\documentclass[12pt]{article}
\usepackage[margin=1in]{geometry}
\usepackage{amsmath}
\parindent=0pt
\pdfinfoomitdate=1
\pdftrailerid{}
\begin{document}
\tableofcontents
\newpage
{\footnotesize\begin{verbatim}
BSD 2-Clause License

Copyright (c) 2024, George Weigt 9634295@gmail.com
All rights reserved.

Redistribution and use in source and binary forms, with or without
modification, are permitted provided that the following conditions are met:

1. Redistributions of source code must retain the above copyright notice, this
   list of conditions and the following disclaimer.

2. Redistributions in binary form must reproduce the above copyright notice,
   this list of conditions and the following disclaimer in the documentation
   and/or other materials provided with the distribution.

THIS SOFTWARE IS PROVIDED BY THE COPYRIGHT HOLDERS AND CONTRIBUTORS "AS IS"
AND ANY EXPRESS OR IMPLIED WARRANTIES, INCLUDING, BUT NOT LIMITED TO, THE
IMPLIED WARRANTIES OF MERCHANTABILITY AND FITNESS FOR A PARTICULAR PURPOSE ARE
DISCLAIMED. IN NO EVENT SHALL THE COPYRIGHT HOLDER OR CONTRIBUTORS BE LIABLE
FOR ANY DIRECT, INDIRECT, INCIDENTAL, SPECIAL, EXEMPLARY, OR CONSEQUENTIAL
DAMAGES (INCLUDING, BUT NOT LIMITED TO, PROCUREMENT OF SUBSTITUTE GOODS OR
SERVICES; LOSS OF USE, DATA, OR PROFITS; OR BUSINESS INTERRUPTION) HOWEVER
CAUSED AND ON ANY THEORY OF LIABILITY, WHETHER IN CONTRACT, STRICT LIABILITY,
OR TORT (INCLUDING NEGLIGENCE OR OTHERWISE) ARISING IN ANY WAY OUT OF THE USE
OF THIS SOFTWARE, EVEN IF ADVISED OF THE POSSIBILITY OF SUCH DAMAGE.
\end{verbatim}}
\newpage

\section{Introduction}

Sassafras is a shell mode program for data analysis.

\subsubsection*{Example}

A die, which may be loaded, is tossed six times.
The observed point values are one to six.
Compute a 95\% confidence interval for the true mean $\mu$
given the observed data.

{\footnotesize\begin{verbatim}
data ;
input y ;
datalines ;
1
2
3
4
5
6
;
proc means clm ;
\end{verbatim}}

The following result is displayed.

{\footnotesize\begin{verbatim}
                    Variable     95% CLM MIN     95% CLM MAX
                    Y                  1.537           5.463
\end{verbatim}}

Here is the same result using R.

{\footnotesize\begin{verbatim}
> y = c(1,2,3,4,5,6)
> t.test(y)

	One Sample t-test

data:  y
t = 4.5826, df = 5, p-value = 0.005934
alternative hypothesis: true mean is not equal to 0
95 percent confidence interval:
 1.536686 5.463314
\end{verbatim}}

\newpage

\section{Data Step}
A data step is used to get data into the program.

\begin{quote}
{\tt
data {\it name} ;

infile "{\it filename}" dlm="{\it delims}" firstobs=$n$ ;

input {\it list} ;

{\it var} = {\it expression} ;

datalines ; {\it data} ;
}
\end{quote}

\subsubsection*{Notes}

\begin{enumerate}
\item
{\it name} is optional.
\item
The {\tt dlm} and {\tt firstobs} settings are optional.
\item
{\it delims} is a sequence of delimiter characters.
The default is tab, comma, and space.
\item
{\it n} is the starting input record number.
Use {\tt firstobs=2} to skip a header in the data file.
\item
{\it list} is a list of variable names separated by spaces.
For each categorical variable place a {\tt\$} after the variable name.
\item
Optional {\it var} = {\it expression} statements
create new vectors in the data set.
\item
The {\tt datalines} statement is followed by observational data.
At the end of the data a semicolon terminates the statement.
\item
For Sassafras installed from the Mac App Store,
data files need to be put in the directory
{\footnotesize\verb$~/Library/Containers/com.gweigt.sassafras/Data/$}
\end{enumerate}

\subsubsection*{Example 1}

The following example is a minimalist data step with in-line data.

{\footnotesize\begin{verbatim}
data ;
input y ;
datalines ;
1
2
3
4
5
6
;
\end{verbatim}}

\subsubsection*{Example 2}

Use \verb$@@$ at the end of an input statement to allow multiple
values on an input line.

{\footnotesize\begin{verbatim}
data ;
input y @@ ;
datalines ;
1 2 3
4 5 6
;
\end{verbatim}}

\subsubsection*{Example 3}

A dollar sign after an input variable indicates that the variable
is categorical instead of numerical.

{\footnotesize\begin{verbatim}
data ;
input trt$ y @@ ;
datalines ;
A 6    A 0    A 2    A 8   A 11
A 4    A 13   A 1    A 8   A 0
B 0    B 2    B 3    B 1   B 18
B 4    B 14   B 9    B 1   B 9
C 13   C 10   C 18   C 5   C 23
C 12   C 5    C 16   C 1   C 20
;
\end{verbatim}}

\subsubsection*{Example 4}

An infile statement is used to read data from a file.
For Sassafras installed from the Mac App Store,
data files need to be put in the directory
{\footnotesize\verb$~/Library/Containers/com.gweigt.sassafras/Data/$}

{\footnotesize\begin{verbatim}
data ;
input color$ x1 x2 x3 x4 x5 x6 x7 x8 x9 x10 x11 y ;
infile "wine.txt" ;
\end{verbatim}}

\subsubsection*{Example 5}

Expressions in a data step create new data vectors.
The following example creates Y2 which is the input
vector Y squared element-wise.

{\footnotesize\begin{verbatim}
data ;
input color$ x1 x2 x3 x4 x5 x6 x7 x8 x9 x10 x11 y ;
infile "wine.txt" ;
y2 = y ** 2 ;
\end{verbatim}}

\newpage

\section{Anova Procedure}

The anova procedure fits a classification model
to data using ordinary least squares.
The response variable must be numeric and the
explanatory variables must be categorical.

\begin{quote}
{\tt
proc anova data={\it name} ;

model {\it y} = {\it list} ;

means {\it list} ;

means {\it list} / lsd ttest alpha={\it value} ;
}
\end{quote}

\subsubsection*{Notes}

\begin{enumerate}
\item
{\tt data=}{\it name} is optional.
The default is data from the most recent data step.
\item
{\it y} is the response variable which must be numeric.
\item
{\it list} is one or more explanatory variables separated by spaces.
The explanatory variables must be categorical.
Interaction terms are specified using the syntax {\tt A*B}.
\item
The means statement can include one or more of the following options.
\begin{quote}
\begin{tabular}{ll}
{\tt lsd} & Compare treatment means using least significance difference \\
{\tt ttest} & Compare treatment means using two sample $t$-test \\
{\tt alpha} & Set the level of significance. Default is $0.05$.
\end{tabular}
\end{quote}
\end{enumerate}

\subsubsection*{Example}

{\footnotesize\begin{verbatim}
data ;
input trt$ y @@ ;
datalines ;
A 6    A 0    A 2    A 8   A 11
A 4    A 13   A 1    A 8   A 0
B 0    B 2    B 3    B 1   B 18
B 4    B 14   B 9    B 1   B 9
C 13   C 10   C 18   C 5   C 23
C 12   C 5    C 16   C 1   C 20
;

proc anova ;
model y = trt ;
means trt / lsd ttest ;
\end{verbatim}}

The following result is displayed.

{\footnotesize\begin{verbatim}
                              Analysis of Variance

    Source     DF     Sum of Squares      Mean Square     F Value     Pr > F
    Model       2       293.60000000     146.80000000        3.98     0.0305
    Error      27       995.10000000      36.85555556                       
    Total      29      1288.70000000                                        

                R-Square     Coeff Var     Root MSE       Y Mean
                0.227826     76.846553     6.070878     7.900000

     Source     DF         Anova SS      Mean Square     F Value     Pr > F
     TRT         2     293.60000000     146.80000000        3.98     0.0305

                                 Mean Response

             TRT      N        Mean Y     95% CI MIN     95% CI MAX
             A       10      5.300000       1.360938       9.239062
             B       10      6.100000       2.160938      10.039062
             C       10     12.300000       8.360938      16.239062

                       Least Significant Difference Test

  TRT    TRT      Delta Y    95% CI MIN    95% CI MAX    t Value    Pr > |t|  
  A      B      -0.800000     -6.370676      4.770676      -0.29      0.7705  
  A      C      -7.000000    -12.570676     -1.429324      -2.58      0.0157 *
  B      A       0.800000     -4.770676      6.370676       0.29      0.7705  
  B      C      -6.200000    -11.770676     -0.629324      -2.28      0.0305 *
  C      A       7.000000      1.429324     12.570676       2.58      0.0157 *
  C      B       6.200000      0.629324     11.770676       2.28      0.0305 *

                               Two Sample t-Test

  TRT    TRT      Delta Y    95% CI MIN    95% CI MAX    t Value    Pr > |t|  
  A      B      -0.800000     -5.922306      4.322306      -0.33      0.7466  
  A      C      -7.000000    -12.664270     -1.335730      -2.60      0.0182 *
  B      A       0.800000     -4.322306      5.922306       0.33      0.7466  
  B      C      -6.200000    -12.467653      0.067653      -2.08      0.0523  
  C      A       7.000000      1.335730     12.664270       2.60      0.0182 *
  C      B       6.200000     -0.067653     12.467653       2.08      0.0523  
\end{verbatim}}

\subsubsection*{Mean response table}

The confidence interval for a treatment mean is computed as follows.
\[
\bar y_i\pm t(1-\alpha/2,dfe)\cdot\sqrt{\frac{MSE}{n_i}}
\]

Recall that $MSE$ is an estimate of model variance.
From the anova table

{\footnotesize\begin{verbatim}
    Error      27       995.10000000      36.85555556
\end{verbatim}}

we obtain
\begin{align*}
MSE&=36.85555556\\
dfe&=27
\end{align*}

Using R, the confidence interval for the mean of treatment A can be
checked as follows.

{\footnotesize\begin{verbatim}
> MSE = 36.8556
> dfe = 27
> t = qt(0.975,dfe)
> 5.3 - t * sqrt(MSE/10)
[1] 1.360934
> 5.3 + t * sqrt(MSE/10)
[1] 9.239066
\end{verbatim}}

\subsubsection*{Least significant difference test}

The least significant difference of two means
$\bar y_i$ and $\bar y_j$ is
\[
LSD_{ij}=t(1-\alpha/2,dfe)\cdot\sqrt{MSE\cdot
\left(\frac{1}{n_i}+\frac{1}{n_j}\right)}
\]

The corresponding confidence interval is
\[
\bar y_i-\bar y_j\pm LSD_{ij}
\]

\subsubsection*{Two sample $t$-test}

The two sample $t$-test is computed as follows.
\begin{align*}
SSE&=\widehat{Var}_i\cdot(n_i-1)+\widehat{Var}_j\cdot(n_j-1)\\
dfe&=n_i+n_j-2\\
MSE&=\frac{SSE}{dfe}\\
SE&=\sqrt{MSE\cdot\left(\frac{1}{n_i}+\frac{1}{n_j}\right)}\\
t^*&=\frac{\bar y_i-\bar y_j}{SE}
\end{align*}

$SSE$ is the sum of squares error recovered from
variance estimates, $dfe$ is the degrees of freedom error, $MSE$
is mean square error, $SE$ is the standard error, and $t^*$ is the
test statistic.
%If $t^*$ is greater than the critical value $t(1-\alpha/2,dfe)$ then
%reject $H_0$ and conclude that the means are different.
The confidence interval is
\[
\bar y_i-\bar y_j\pm t(1-\alpha/2,dfe)\cdot SE
\]

The null hypothesis is that the two treatment means are equal.
\[
H_0:\bar y_i=\bar y_j
\]

If $|t^*|$ is greater than the critical value $t(1-\alpha/2,dfe)$,
or equivalently, if the confidence interval does not cross zero,
then reject $H_0$ and conclude that the treatment means are not equal.
The following R session uses the above equations
to duplicate the Sassafras result for
treatments A and B.

{\footnotesize\begin{verbatim}
> YA = c(6,0,2,8,11,4,13,1,8,0)
> YB = c(0,2,3,1,18,4,14,9,1,9)
> sse = var(YA) * (length(YA) - 1) + var(YB) * (length(YB) - 1)
> dfe = length(YA) + length(YB) - 2
> mse = sse / dfe
> se = sqrt(mse * (1 / length(YA) + 1 / length(YB)))
> t = (mean(YA) - mean(YB)) / se
> mean(YA) - mean(YB) - qt(0.975,dfe) * se
[1] -5.922307
> mean(YA) - mean(YB) + qt(0.975,dfe) * se
[1] 4.322307
> 2 * (1 - pt(abs(t),dfe))
[1] 0.746606
\end{verbatim}}

The same result is obtained with the t-test function.

{\footnotesize\begin{verbatim}
> t.test(YA,YB,var.equal=TRUE)

	Two Sample t-test

data:  YA and YB
t = -0.3281, df = 18, p-value = 0.7466
alternative hypothesis: true difference in means is not equal to 0
95 percent confidence interval:
 -5.922307  4.322307
sample estimates:
mean of x mean of y 
      5.3       6.1 
\end{verbatim}}

\newpage

\section{Means Procedure}

The means procedure prints statistics about a data set.

\begin{quote}
{\tt
proc means data={\it name} alpha={\it value} maxdec=$n$ {\it stats} ;

var {\it list} ;

class {\it list} ;
}
\end{quote}

\subsubsection*{Notes}

\begin{enumerate}
\item
The settings that follow the {\tt means} keyword are optional.
The settings can appear in any order.
\item
If {\tt data} is not specified then the default is data from
the most recent data step.
\item
{\tt alpha} sets the level of significance.
The default is $0.05$.
\item
{\tt maxdec} sets the decimal precision in the output.
$n$ ranges from 0 to 8.
The default is 3.
\item
{\it stats} is a list of statistics keywords
from the following table.
\begin{quote}
\begin{tabular}{ll}
{\tt clm} & Confidence limits of the mean\\
{\tt max} & Maximum value\\
{\tt mean} & Mean value\\
{\tt min} & Minimum value\\
{\tt n} & Number of observations\\
{\tt range} & max $-$ min\\
{\tt std} & Standard deviation $s$\\
{\tt stddev} & Another keyword for $s$\\
{\tt stderr} & Standard error $s/\sqrt n$\\
{\tt var} & Variance $s^2$
\end{tabular}
\end{quote}
If {\it stats} is not specified then the default list is
{\tt n mean std min max}.
\item
The optional {\tt var} statement specifies which variables to print.
The default is all variables.
Variable names in {\it list} are separated by spaces.
\item
The optional {\tt class} statement prints statistics for each level
of the categorical variables in {\it list}.
Variable names in {\it list} are separated by spaces.
\end{enumerate}

\subsubsection*{Example 1}

The following example reads in the wine\footnote{
P. Cortez, A. Cerdeira, F. Almeida, T. Matos and J. Reis.
{\it Modeling wine preferences by data mining from physicochemical properties.}
In Decision Support Systems, Elsevier, 47(4):547-553, 2009.}
data set and shows the default action of proc means.

{\footnotesize\begin{verbatim}
data wine ;
input color$ x1 x2 x3 x4 x5 x6 x7 x8 x9 x10 x11 y ;
infile "wine.txt" ;

proc means ;
\end{verbatim}}

The following result is displayed.

{\footnotesize\begin{verbatim}
Variable        N        Mean     Std Dev     Minimum     Maximum
X1           6497       7.215       1.296       3.800      15.900
X2           6497       0.340       0.165       0.080       1.580
X3           6497       0.319       0.145       0.000       1.660
X4           6497       5.443       4.758       0.600      65.800
X5           6497       0.056       0.035       0.009       0.611
X6           6497      30.525      17.749       1.000     289.000
X7           6497     115.745      56.522       6.000     440.000
X8           6497       0.995       0.003       0.987       1.039
X9           6497       3.219       0.161       2.720       4.010
X10          6497       0.531       0.149       0.220       2.000
X11          6497      10.492       1.193       8.000      14.900
Y            6497       5.818       0.873       3.000       9.000
\end{verbatim}}

\subsubsection*{Example 2}

The following example adds a var statement to show Y by itself.
Also, the desired statistics are specified.

{\footnotesize\begin{verbatim}
data wine ;
input color$ x1 x2 x3 x4 x5 x6 x7 x8 x9 x10 x11 y ;
infile "wine.txt" ;

proc means n mean clm ;
var y ;
\end{verbatim}}

The following result is displayed.

{\footnotesize\begin{verbatim}
Variable        N      Mean     95% CLM MIN     95% CLM MAX
Y            6497     5.818           5.797           5.840
\end{verbatim}}

\subsubsection*{Example 3}

The following example adds a class statement
to show statistics for each wine color.

{\footnotesize\begin{verbatim}
data wine ;
input color$ x1 x2 x3 x4 x5 x6 x7 x8 x9 x10 x11 y ;
infile "wine.txt" ;

proc means n mean clm ;
var y ;
class color ;
\end{verbatim}}

The following result is displayed.

{\footnotesize\begin{verbatim}
COLOR     Variable        N      Mean     95% CLM MIN     95% CLM MAX
red       Y            1599     5.636           5.596           5.676
white     Y            4898     5.878           5.853           5.903
\end{verbatim}}

\newpage

\section{Print Procedure}

The print procedure prints data in a data set.

\begin{quote}
{\tt
proc print data={\it name} ;

var {\it list} ;
}
\end{quote}

\subsubsection*{Notes}

\begin{enumerate}
\item
{\tt data=}{\it name} is optional.
The default is data from the most recent data step.
\item
The optional {\tt var} statement specifies which variables
to print.
The default is all variables.
Variable names in {\it list} are separated by spaces.
\end{enumerate}

\subsubsection*{Example}

The following example reads a data set
and prints it.

{\footnotesize\begin{verbatim}
data ;
input trt$ y @@ ;
datalines ;
A 6    A 0    A 2    A 8   A 11
A 4    A 13   A 1    A 8   A 0
B 0    B 2    B 3    B 1   B 18
B 4    B 14   B 9    B 1   B 9
;

proc print ;
\end{verbatim}}

The following result is displayed.

{\footnotesize\begin{verbatim}
                               Obs     TRT      Y
                                 1     A        6
                                 2     A        0
                                 3     A        2
                                 4     A        8
                                 5     A       11
                                 6     A        4
                                 7     A       13
                                 8     A        1
                                 9     A        8
                                10     A        0
                                11     B        0
                                12     B        2
                                13     B        3
                                14     B        1
                                15     B       18
                                16     B        4
                                17     B       14
                                18     B        9
                                19     B        1
                                20     B        9
\end{verbatim}}

\newpage

\section{Reg Procedure}

The reg procedure fits a linear model to data
using ordinary least squares.
The response variable must be numeric.
For models with no intercept, anova results will differ from R.
This is because R switches to uncorrected sums of squares
for models with no intercept.

\begin{quote}
{\tt
proc reg data={\it name} ;

model {\it y} = {\it list} ;

model {\it y} = {\it list} / noint ;
}
\end{quote}

\subsubsection*{Notes}

\begin{enumerate}
\item
{\tt data=}{\it name} is optional.
The default is data from the most recent data step.
\item
{\it y} is the response variable which must be numeric.
\item
{\it list} is a list of explanatory variables separated by spaces.
If functions of explanatory variables are required then
they must be defined in the data step.
\item
The {\tt noint} option fits a linear model with no intercept term.
\end{enumerate}

\subsubsection*{Example 1}

The following example reads in the wine data set and fits
a linear model with no intercept term.

{\footnotesize\begin{verbatim}
data ;
input color$ x1 x2 x3 x4 x5 x6 x7 x8 x9 x10 x11 y ;
infile "wine.txt" ;

proc reg ;
model y = color x1 / noint ;
\end{verbatim}}

The following result is displayed.

{\footnotesize\begin{verbatim}
                              Analysis of Variance

                DF     Sum of Squares     Mean Square     F Value     Pr > F
    Model        2           72.79210        36.39605       48.42     0.0000
    Error     6494         4880.89360         0.75160                       
    Total     6496         4953.68570                                       

              Root MSE            0.86695     R-Square     0.0147
              Dependent Mean      5.81838     Adj R-Sq     0.0144
              Coeff Var          14.90018                        

                              Parameter Estimates

                         Estimate     Std Err     t Value     Pr > |t|
         COLOR red        5.77309     0.08194       70.45       0.0000
         COLOR white      5.99084     0.06628       90.39       0.0000
         X1              -0.01647     0.00950       -1.73       0.0829
\end{verbatim}}

\subsubsection*{Example 2}

The following exercise is from {\it Econometrics}\footnote{
Hansen, Bruce E. {\it Econometrics.}
www.ssc.wisc.edu/$\sim$bhansen}.
Using data from a 1963 paper by Marc Nerlove,
estimate parameters for the model
\[
\log({\rm COST})=\beta_0+\beta_1\log({\rm KWH})
+\beta_2\log({\rm PL})+\beta_3\log({\rm PF})+\beta_4\log({\rm PK})
+\varepsilon
\]
where COST is production cost,
KWH is kilowatt hours,
PL is price of labor,
PF is price of fuel,
and PK is price of capital.

{\footnotesize\begin{verbatim}
data ;
infile "nerlove.txt" ;
input COST KWH PL PF PK ;
LCOST = log(COST) ;
LKWH = log(KWH) ;
LPL = log(PL) ;
LPF = log(PF) ;
LPK = log(PK) ;

proc reg ;
model LCOST = LKWH LPL LPF LPK ;
\end{verbatim}}

The following result is displayed.

{\footnotesize\begin{verbatim}
                              Analysis of Variance

               DF     Sum of Squares     Mean Square     F Value     Pr > F
    Model       4          269.51482        67.37870      437.69     0.0000
    Error     140           21.55201         0.15394                       
    Total     144          291.06683                                       

              Root MSE            0.39236     R-Square     0.9260
              Dependent Mean      1.72466     Adj R-Sq     0.9238
              Coeff Var          22.74969                        

                              Parameter Estimates

                        Estimate     Std Err     t Value     Pr > |t|
          Intercept     -3.52650     1.77437       -1.99       0.0488
          LKWH           0.72039     0.01747       41.24       0.0000
          LPL            0.43634     0.29105        1.50       0.1361
          LPF            0.42652     0.10037        4.25       0.0000
          LPK           -0.21989     0.33943       -0.65       0.5182
\end{verbatim}}

The following code can be pasted into R to obtain a similar result.

{\footnotesize\begin{verbatim}
d = read.table("nerlove.txt")
lcost = log(d[,1])
lkwh = log(d[,2])
lpl = log(d[,3])
lpf = log(d[,4])
lpk = log(d[,5])
m = lm(lcost ~ lkwh + lpl + lpf + lpk)
summary(m)
\end{verbatim}}

The following result is displayed in R.

{\footnotesize\begin{verbatim}
Call:
lm(formula = lcost ~ lkwh + lpl + lpf + lpk)

Residuals:
     Min       1Q   Median       3Q      Max 
-0.97784 -0.23817 -0.01372  0.16031  1.81751 

Coefficients:
            Estimate Std. Error t value Pr(>|t|)    
(Intercept) -3.52650    1.77437  -1.987   0.0488 *  
lkwh         0.72039    0.01747  41.244  < 2e-16 ***
lpl          0.43634    0.29105   1.499   0.1361    
lpf          0.42652    0.10037   4.249 3.89e-05 ***
lpk         -0.21989    0.33943  -0.648   0.5182    
---
Signif. codes:  0 `***' 0.001 `**' 0.01 `*' 0.05 `.' 0.1 ` ' 1

Residual standard error: 0.3924 on 140 degrees of freedom
Multiple R-squared:  0.926,	Adjusted R-squared:  0.9238 
F-statistic: 437.7 on 4 and 140 DF,  p-value: < 2.2e-16
\end{verbatim}}

\subsubsection*{Example 3}

The following model uses the ``trees'' data set from R.

{\footnotesize\begin{verbatim}
data ;
input Girth Height Volume ;
LG = log(Girth) ;
LH = log(Height) ;
LV = log(Volume) ;
datalines ;
  8.3     70   10.3
  8.6     65   10.3
  8.8     63   10.2
 10.5     72   16.4
 10.7     81   18.8
 10.8     83   19.7
 11.0     66   15.6
 11.0     75   18.2
 11.1     80   22.6
 11.2     75   19.9
 11.3     79   24.2
 11.4     76   21.0
 11.4     76   21.4
 11.7     69   21.3
 12.0     75   19.1
 12.9     74   22.2
 12.9     85   33.8
 13.3     86   27.4
 13.7     71   25.7
 13.8     64   24.9
 14.0     78   34.5
 14.2     80   31.7
 14.5     74   36.3
 16.0     72   38.3
 16.3     77   42.6
 17.3     81   55.4
 17.5     82   55.7
 17.9     80   58.3
 18.0     80   51.5
 18.0     80   51.0
 20.6     87   77.0
;

proc reg ;
model LV = LG LH ;
\end{verbatim}}

The following result is displayed.

{\footnotesize\begin{verbatim}
                              Analysis of Variance

    Source     DF     Sum of Squares     Mean Square     F Value     Pr > F
    Model       2         1.53213547      0.76606773      613.19     0.0000
    Error      28         0.03498056      0.00124931                       
    Total      30         1.56711603                                       

               Root MSE           0.03535     R-Square     0.9777
               Dependent Mean     1.42133     Adj R-Sq     0.9761
               Coeff Var          2.48679                        

                              Parameter Estimates

         Parameter       Estimate     Std Err     t Value     Pr > |t|
         (Intercept)     -2.88007     0.34734       -8.29       0.0000
         log(Girth)       1.98265     0.07501       26.43       0.0000
         log(Height)      1.11712     0.20444        5.46       0.0000
\end{verbatim}}

Let us see if the above parameters correspond to the volume of a cone
given by
$$
V=\frac{\pi}{12}d^2h
$$
where $d$ is the diameter (girth) and $h$ is the height of the cone.
The model from the regression is
$$
\log V=-2.88+1.98\log d+1.12\log h
$$

Take the antilog of both sides and obtain
$$
V=0.00132\times d^{1.98}\times h^{1.12}
$$

The exponents resemble the volume formula but the overall coefficient 0.00132
is two orders of magnitude smaller than $\pi/12\approx0.262$.
It turns out the discrepancy is due to the units of measure.
Girth is measured in inches while height and volume are measured in feet.
To convert girth from inches to feet requires a factor of $1/12$.
Hence the leading coefficient should be
$$
\frac{\pi}{12}\times\frac{1}{144}\approx0.00182
$$
which is in the ballpark of 0.00132 from the regression model.

\bigskip

Let us compare the Reg results to R.
The following block of code can be pasted directly into the R shell prompt.

{\footnotesize\begin{verbatim}
d=log10(trees[,1])
h=log10(trees[,2])
V=log10(trees[,3])
m=lm(V~d+h)
summary(m)
\end{verbatim}}

This is the R result, which matches Reg.

{\footnotesize\begin{verbatim}
Coefficients:
            Estimate Std. Error t value Pr(>|t|)    
(Intercept) -2.88007    0.34734  -8.292 5.06e-09 ***
d            1.98265    0.07501  26.432  < 2e-16 ***
h            1.11712    0.20444   5.464 7.81e-06 ***
---
Signif. codes:  0 `***' 0.001 `**' 0.01 `*' 0.05 `.' 0.1 ` ' 1

Residual standard error: 0.03535 on 28 degrees of freedom
Multiple R-squared:  0.9777,	Adjusted R-squared:  0.9761 
F-statistic: 613.2 on 2 and 28 DF,  p-value: < 2.2e-16
\end{verbatim}}

\newpage

\section{Review}

\subsubsection*{Analysis of Variance}

The components of an analysis of variance table are computed as follows.

{\footnotesize\begin{center}\begin{tabular}{|l|c|c|c|c|c|}
\hline
& DF & SS & Mean Square & $F$-value & $p$-value\\
\hline
Model
& $p-1$
& $SSR$
& $MSR=SSR/(p-1)$
& $F^*=MSR/MSE$
& $1-F(F^*,p-1,n-p)$\\
Error
& $n-p$
& $SSE$
& $MSE=SSE/(n-p)$
& &\\
Total
& $n-1$
& $SST$
& & &\\
\hline
\end{tabular}\end{center}}

\bigskip

In the table, $n$ is the number of observations and $p$ is the
number of model parameters including the intercept term
if there is one.
The sums of squares are computed as follows.
\begin{align*}
SSR&=\sum(\hat y_i-\bar y)^2\\
SSE&=\sum(y_i-\hat y_i)^2\\
SST&=\sum(y_i-\bar y)^2
%\\
%R^2&=1-\frac{SSE}{SST}\\
%CV&=100\times\frac{\sqrt{MSE}}{\bar y}\\
%MSE&=\hat\sigma^2
\end{align*}

Recall that $MSE$ is an estimate of model variance.
\[
MSE=\hat\sigma^2
\]

A simple way to model the response variable is to use the average $\bar y$.
The $p$-value above indicates whether or not the regression model
is better than $\bar y$.
The null hypothesis is that the regression model is no better
than the average, that is
\[
H_0:SST=SSE
\]

The test for $H_0$ is known as an omnibus test because an
equivalent hypothesis is
\[
H_0:\beta_1=\beta_2=\cdots=\beta_{p-1}=0
\]

Under $H_0$ we have $SSR=0$ hence another
equivalent hypothesis is
\[
H_0:F^*=0
\]

The test statistic $F^*$ is used
because it has a well-known distribution.
Recall that the $p$-value is (loosely) the probability
that $H_0$ is true.
Hence for small $p$-values, reject $H_0$ and
conclude that the regression model is better than $\bar y$.

\subsubsection*{Confidence interval of the mean}
The confidence interval of the mean is
\[
\bar x\pm t_{1-\alpha/2, n-1}\frac{s}{\sqrt n}
\]
where $\bar x$ is the observed mean,
$s$ is the observed standard deviation,
$n$ is the number of observations,
and $t_{1-\alpha/2, n-1}$ is the quantile function.
%(Let $F^{-1}(y)$ be the quantile function.
%Then for $t=F^{-1}(y)$ we have $F(t)=1-\alpha/2$.)
In R, the confidence interval of the mean of \verb$1:10$
can be computed as follows.

{\footnotesize\begin{verbatim}
> x = 1:10
> n = length(x)
> alpha = 0.05
> mean(x) - qt(1-alpha/2,n-1) * sd(x)/sqrt(n)
[1] 3.334149
> mean(x) + qt(1-alpha/2,n-1) * sd(x)/sqrt(n)
[1] 7.665851
\end{verbatim}}

Alternatively, the \verb$t.test$ function can be used.

{\footnotesize\begin{verbatim}
> t.test(1:10)

	One Sample t-test

data:  1:10
t = 5.7446, df = 9, p-value = 0.0002782
alternative hypothesis: true mean is not equal to 0
95 percent confidence interval:
 3.334149 7.665851
sample estimates:
mean of x 
      5.5 
\end{verbatim}}

Recall that the quantile function
is the inverse of the cumulative
distribution function.
Let $F$ be the cumulative distribution function.
Then
\[
F(t_{1-\alpha/2,n-1})=1-\alpha/2
\]

For example, in R we have

{\footnotesize\begin{verbatim}
> t = qt(0.975,8)
> t
[1] 2.306004
> pt(t,8)
[1] 0.975
\end{verbatim}}

\newpage

\section{Anova results}

Consider the following anova program and its output.
Note that the least significant difference test has more power (more stars) than the $t$-test.

{\footnotesize\begin{verbatim}
data ;
input trt$ y @@ ;
datalines ;
A 6    A 0    A 2    A 8   A 11
A 4    A 13   A 1    A 8   A 0
B 0    B 2    B 3    B 1   B 18
B 4    B 14   B 9    B 1   B 9
C 13   C 10   C 18   C 5   C 23
C 12   C 5    C 16   C 1   C 20
;

proc anova ;
model y = trt ;
means trt / lsd ttest ;
\end{verbatim}}

{\footnotesize\begin{verbatim}
                              Analysis of Variance

    Source     DF     Sum of Squares      Mean Square     F Value     Pr > F
    Model       2       293.60000000     146.80000000        3.98     0.0305
    Error      27       995.10000000      36.85555556                       
    Total      29      1288.70000000                                        

                R-Square     Coeff Var     Root MSE       Y Mean
                0.227826     76.846553     6.070878     7.900000

     Source     DF         Anova SS      Mean Square     F Value     Pr > F
     TRT         2     293.60000000     146.80000000        3.98     0.0305

                                 Mean Response

             TRT      N        Mean Y     95% CI MIN     95% CI MAX
             A       10      5.300000       1.360937       9.239063
             B       10      6.100000       2.160937      10.039063
             C       10     12.300000       8.360937      16.239063

                       Least Significant Difference Test

  TRT    TRT      Delta Y    95% CI MIN    95% CI MAX    t Value    Pr > |t|  
  A      B      -0.800000     -6.370677      4.770677      -0.29      0.7705  
  A      C      -7.000000    -12.570677     -1.429323      -2.58      0.0157 *
  B      A       0.800000     -4.770677      6.370677       0.29      0.7705  
  B      C      -6.200000    -11.770677     -0.629323      -2.28      0.0305 *
  C      A       7.000000      1.429323     12.570677       2.58      0.0157 *
  C      B       6.200000      0.629323     11.770677       2.28      0.0305 *

                               Two Sample t-Test

  TRT    TRT      Delta Y    95% CI MIN    95% CI MAX    t Value    Pr > |t|  
  A      B      -0.800000     -5.922307      4.322307      -0.33      0.7466  
  A      C      -7.000000    -12.664270     -1.335730      -2.60      0.0182 *
  B      A       0.800000     -4.322307      5.922307       0.33      0.7466  
  B      C      -6.200000    -12.467653      0.067653      -2.08      0.0523  
  C      A       7.000000      1.335730     12.664270       2.60      0.0182 *
  C      B       6.200000     -0.067653     12.467653       2.08      0.0523  
\end{verbatim}}

Let us take a closer look at the analysis of variance table.

{\footnotesize\begin{verbatim}
                              Analysis of Variance

    Source     DF     Sum of Squares      Mean Square     F Value     Pr > F
    Model       2       293.60000000     146.80000000        3.98     0.0305
    Error      27       995.10000000      36.85555556                       
    Total      29      1288.70000000
\end{verbatim}}

This is how the table values are computed where $n$ is the number of observations
and $p$ is the number of model parameters.

{\footnotesize\begin{center}\begin{tabular}{lccccc}
Source & DF & Sum of Squares & Mean Square & $F$-value & $p$-value
\\
Model
& $p-1$
& SSR
& $\text{MSR}=\text{SSR}/(p-1)$
& $F^*=\text{MSR}/\text{MSE}$
& $1-F(F^*,p-1,n-p)$\\
Error
& $n-p$
& SSE
& $\text{MSE}=\text{SSE}/(n-p)$
& &\\
Total
& $n-1$
& SST
& & &
\end{tabular}\end{center}}

For the following sum of squares calculations, $y$ are observed values and $\hat y$ are predicted values.
\begin{align*}
\text{SSR}&=\sum(\hat y_i-\bar y)^2=\text{SST}-\text{SSE}
\\
\text{SSE}&=\sum(y_i-\hat y_i)^2
\\
\text{SST}&=\sum(y_i-\bar y)^2
\end{align*}

%Recall that MSE is an estimate of model variance.
%\begin{equation*}
%\text{MSE}=\frac{\text{SSE}}{n-p}=\hat\sigma^2
%\end{equation*}

The $p$-value in the anova table is used for checking that the regression model
is better than the mean $\bar y$.
The null hypothesis is that the model is no better than the mean, that is
\begin{equation*}
H_0:\text{SSE}=\text{SST}
\end{equation*}

%The test for $H_0$ is known as an omnibus test because an
%equivalent hypothesis is that all of the model parameters are zero.
%\[
%H_0:\beta_1=\beta_2=\cdots=\beta_{p-1}=0
%\]

Under $H_0$ we have $\text{SSR}=0$ hence $\text{MSR}=0$ and
\begin{equation*}
H_0:F^*=0
\end{equation*}

%The test statistic $F^*$ is used because it has a well-known distribution, i.e., the $F$-distribution.
Recall that the $p$-value is (loosely) the probability that $H_0$ is true.
Hence for small $p$-values, reject $H_0$ and conclude that the regression model is better than the mean.

Let us take a closer look at the mean response table.

{\footnotesize\begin{verbatim}
                                 Mean Response

             TRT      N        Mean Y     95% CI MIN     95% CI MAX
             A       10      5.300000       1.360937       9.239063
             B       10      6.100000       2.160937      10.039063
             C       10     12.300000       8.360937      16.239063
\end{verbatim}}

Recall that the confidence interval for a treatment mean is
\begin{equation*}
\bar y\pm t(1-\alpha/2,\text{dfe})\times\text{SE},
\quad
\text{SE}=\sqrt{\frac{\text{MSE}}{n}}
\end{equation*}
where SE is standard error and MSE (mean square error) is estimated model variance.
From the analysis of variance table at the top of the output we have

{\footnotesize\begin{verbatim}
    Source     DF     Sum of Squares      Mean Square
    Error      27       995.10000000      36.85555556
\end{verbatim}}

Hence
\begin{equation*}
\text{dfe}=27,
\quad
\text{MSE}=36.85555556
\end{equation*}

The confidence interval for the mean of treatment A can be checked by typing the following into R.

{\footnotesize\begin{verbatim}
ybar = 5.3
n = 10
MSE = 36.85555556
dfe = 27
alpha = 0.05
SE = sqrt(MSE / n)
t = qt(1 - alpha/2, dfe) * SE
ybar - t
ybar + t
\end{verbatim}}

R prints the following results.

{\footnotesize\begin{verbatim}
[1] 1.360937
[1] 9.239063
\end{verbatim}}

The R results match the mean response table for treatment A.

{\footnotesize\begin{verbatim}
             TRT      N        Mean Y     95% CI MIN     95% CI MAX
             A       10      5.300000       1.360937       9.239063
\end{verbatim}}

Let us take a closer look at the least significant difference table.

{\footnotesize\begin{verbatim}
                       Least Significant Difference Test

  TRT    TRT      Delta Y    95% CI MIN    95% CI MAX    t Value    Pr > |t|  
  A      B      -0.800000     -6.370677      4.770677      -0.29      0.7705  
\end{verbatim}}

The least significant difference of two treatment means $\bar y_A$ and $\bar y_B$ is
\begin{equation*}
\text{LSD}=t(1-\alpha/2,\text{dfe})\times\text{SE},
\quad
\text{SE}=\sqrt{\text{MSE}\times\left(\frac{1}{n_A}+\frac{1}{n_B}\right)}
\end{equation*}

The corresponding confidence interval is
\begin{equation*}
(\bar y_A-\bar y_B)\pm\text{LSD}
\end{equation*}

The confidence interval in the LSD table can be checked by typing the following into R.

{\footnotesize\begin{verbatim}
ybarA = 5.3
ybarB = 6.1
nA = 10
nB = 10
MSE = 36.85555556
dfe = 27
alpha = 0.05
SE = sqrt(MSE * (1/nA + 1/nB))
LSD = qt(1 - alpha/2, dfe) * SE
ybarA - ybarB - LSD
ybarA - ybarB + LSD
\end{verbatim}}

R prints the following results.

{\footnotesize\begin{verbatim}
[1] -6.370677
[1] 4.770677
\end{verbatim}}

The R results match the confidence interval in the LSD table.

{\footnotesize\begin{verbatim}
  TRT    TRT      Delta Y    95% CI MIN    95% CI MAX    t Value    Pr > |t|  
  A      B      -0.800000     -6.370677      4.770677      -0.29      0.7705  
\end{verbatim}}

Let us take a closer look at the $t$-test table.

{\footnotesize\begin{verbatim}
                               Two Sample t-Test

  TRT    TRT      Delta Y    95% CI MIN    95% CI MAX    t Value    Pr > |t|  
  A      B      -0.800000     -5.922307      4.322307      -0.33      0.7466  
\end{verbatim}}

The $t$-test confidence interval is
\begin{equation*}
(\bar y_A-\bar y_B)\pm t(1-\alpha/2,\text{dfe})\times\text{SE}
\end{equation*}
where
\begin{equation*}
\text{SE}=\sqrt{\frac{\text{SSE}}{\text{dfe}}\times\left(\frac{1}{n_A}+\frac{1}{n_B}\right)},
\quad
\text{SSE}=\sum(y_A-\bar y_A)^2+\sum(y_B-\bar y_B)^2
\end{equation*}
and
\begin{equation*}
\text{dfe}=n_A+n_B-2
\end{equation*}

The confidence interval can be checked by typing the following into R.

{\footnotesize\begin{verbatim}
yA = c(6,0,2,8,11,4,13,1,8,0)
yB = c(0,2,3,1,18,4,14,9,1,9)
nA = length(yA)
nB = length(yB)
dfe = nA + nB - 2
SSE = var(yA) * (nA - 1) + var(yB) * (nB - 1)
MSE = SSE / dfe
SE = sqrt(MSE * (1/nA + 1/nB))
alpha = 0.05
t = qt(1 - alpha/2, dfe) * SE
mean(yA) - mean(yB) - t
mean(yA) - mean(yB) + t
\end{verbatim}}

R prints the following result which matches the above $t$-test table.

{\footnotesize\begin{verbatim}
[1] -5.922307
[1] 4.322307
\end{verbatim}}

R's $t$-test function gives the same result.

{\footnotesize\begin{verbatim}
t.test(yA,yB,var.equal=TRUE)

	Two Sample t-test

data:  yA and yB
t = -0.32812, df = 18, p-value = 0.7466
alternative hypothesis: true difference in means is not equal to 0
95 percent confidence interval:
 -5.922307  4.322307
\end{verbatim}}

\end{document}
