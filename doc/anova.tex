\documentclass[12pt]{article}
\usepackage{amsmath}
\parindent=0pt
\begin{document}

Consider the following anova program and its output.
Note that the least significant difference test has more power than the $t$-test.

{\scriptsize\begin{verbatim}
data ;
input trt $ y @@ ;
datalines ;
A 6    A 0    A 2    A 8   A 11
A 4    A 13   A 1    A 8   A 0
B 0    B 2    B 3    B 1   B 18
B 4    B 14   B 9    B 1   B 9
C 13   C 10   C 18   C 5   C 23
C 12   C 5    C 16   C 1   C 20
;

proc anova ;
model y = trt ;
means trt / lsd ttest ;
\end{verbatim}}

{\scriptsize\begin{verbatim}
                              Analysis of Variance

    Source     DF     Sum of Squares      Mean Square     F Value     Pr > F
    Model       2       293.60000000     146.80000000        3.98     0.0305
    Error      27       995.10000000      36.85555556                       
    Total      29      1288.70000000                                        

                R-Square     Coeff Var     Root MSE       Y Mean
                0.227826     76.846553     6.070878     7.900000

     Source     DF         Anova SS      Mean Square     F Value     Pr > F
     TRT         2     293.60000000     146.80000000        3.98     0.0305

                                 Mean Response

             TRT      N        Mean Y     95% CI MIN     95% CI MAX
             A       10      5.300000       1.360937       9.239063
             B       10      6.100000       2.160937      10.039063
             C       10     12.300000       8.360937      16.239063

                       Least Significant Difference Test

  TRT    TRT      Delta Y    95% CI MIN    95% CI MAX    t Value    Pr > |t|  
  A      B      -0.800000     -6.370677      4.770677      -0.29      0.7705  
  A      C      -7.000000    -12.570677     -1.429323      -2.58      0.0157 *
  B      A       0.800000     -4.770677      6.370677       0.29      0.7705  
  B      C      -6.200000    -11.770677     -0.629323      -2.28      0.0305 *
  C      A       7.000000      1.429323     12.570677       2.58      0.0157 *
  C      B       6.200000      0.629323     11.770677       2.28      0.0305 *

                               Two Sample t-Test

  TRT    TRT      Delta Y    95% CI MIN    95% CI MAX    t Value    Pr > |t|  
  A      B      -0.800000     -5.922307      4.322307      -0.33      0.7466  
  A      C      -7.000000    -12.664270     -1.335730      -2.60      0.0182 *
  B      A       0.800000     -4.322307      5.922307       0.33      0.7466  
  B      C      -6.200000    -12.467653      0.067653      -2.08      0.0523  
  C      A       7.000000      1.335730     12.664270       2.60      0.0182 *
  C      B       6.200000     -0.067653     12.467653       2.08      0.0523  
\end{verbatim}}

Let us take a closer look at the mean response table.

{\scriptsize\begin{verbatim}
                                 Mean Response

             TRT      N        Mean Y     95% CI MIN     95% CI MAX
             A       10      5.300000       1.360937       9.239063
             B       10      6.100000       2.160937      10.039063
             C       10     12.300000       8.360937      16.239063
\end{verbatim}}

Recall that the confidence interval for a treatment mean is
\begin{equation*}
\bar y\pm t(1-\alpha/2,\text{dfe})\times\text{SE},
\quad
\text{SE}=\sqrt{\frac{\text{MSE}}{n}}
\end{equation*}

Recall that SE is standard error and MSE (mean square error) is estimated model variance.
From the analysis of variance table at the top of the output we have

{\scriptsize\begin{verbatim}
    Source     DF     Sum of Squares      Mean Square
    Error      27       995.10000000      36.85555556
\end{verbatim}}

Hence
\begin{equation*}
\text{dfe}=27,
\quad
\text{MSE}=36.8556
\end{equation*}

The confidence interval for the mean of treatment A can be checked by typing the following into R.

{\scriptsize\begin{verbatim}
ybar = 5.3
n = 10
MSE = 36.8556
dfe = 27
alpha = 0.05
SE = sqrt(MSE / n)
t = qt(1 - alpha/2, dfe) * SE
ybar - t
ybar + t
\end{verbatim}}

R prints the following results.

{\scriptsize\begin{verbatim}
[1] 1.360934
[1] 9.239066
\end{verbatim}}

The R results match the mean response table for treatment A.

{\scriptsize\begin{verbatim}
             TRT      N        Mean Y     95% CI MIN     95% CI MAX
             A       10      5.300000       1.360937       9.239063
\end{verbatim}}

\newpage

Let us take a closer look at the first line of the least significant difference table.

{\scriptsize\begin{verbatim}
                       Least Significant Difference Test

  TRT    TRT      Delta Y    95% CI MIN    95% CI MAX    t Value    Pr > |t|  
  A      B      -0.800000     -6.370677      4.770677      -0.29      0.7705  
\end{verbatim}}

The least significant difference of two treatment means $\bar y_A$ and $\bar y_B$ is
\begin{equation*}
\text{LSD}=t(1-\alpha/2,\text{dfe})\times\text{SE},
\quad
\text{SE}=\sqrt{\text{MSE}\times\left(\frac{1}{n_A}+\frac{1}{n_B}\right)}
\end{equation*}

The corresponding confidence interval is
\begin{equation*}
\bar y_A-\bar y_B\pm\text{LSD}
\end{equation*}

The confidence interval in the LSD table can be checked by typing the following into R.

{\scriptsize\begin{verbatim}
ybarA = 5.3
ybarB = 6.1
nA = 10
nB = 10
MSE = 36.8556
dfe = 27
alpha = 0.05
SE = sqrt(MSE * (1/nA + 1/nB))
LSD = qt(1 - alpha/2, dfe) * SE
ybarA - ybarB - LSD
ybarA - ybarB + LSD
\end{verbatim}}

R prints the following results.

{\scriptsize\begin{verbatim}
[1] -6.37068
[1] 4.77068
\end{verbatim}}

The R results match the confidence interval in the LSD table.

{\scriptsize\begin{verbatim}
  TRT    TRT      Delta Y    95% CI MIN    95% CI MAX    t Value    Pr > |t|  
  A      B      -0.800000     -6.370677      4.770677      -0.29      0.7705  
\end{verbatim}}

\end{document}

The two sample $t$-test is computed as follows.
\begin{align*}
SSE&=\widehat{Var}_i\cdot(n_i-1)+\widehat{Var}_j\cdot(n_j-1)\\
dfe&=n_i+n_j-2\\
MSE&=\frac{SSE}{dfe}\\
SE&=\sqrt{MSE\cdot\left(\frac{1}{n_i}+\frac{1}{n_j}\right)}\\
t^*&=\frac{\bar y_i-\bar y_j}{SE}
\end{align*}

$SSE$ is the sum of squares error recovered from
variance estimates, $dfe$ is the degrees of freedom error, $MSE$
is mean square error, $SE$ is the standard error, and $t^*$ is the
test statistic.
%If $t^*$ is greater than the critical value $t(1-\alpha/2,dfe)$ then
%reject $H_0$ and conclude that the means are different.
The confidence interval is
\[
\bar y_i-\bar y_j\pm t(1-\alpha/2,dfe)\cdot SE
\]
The null hypothesis is that the two treatment means are equal.
\[
H_0:\bar y_i=\bar y_j
\]
If $|t^*|$ is greater than the critical value $t(1-\alpha/2,dfe)$,
or equivalently, if the confidence interval does not cross zero,
then reject $H_0$ and conclude that the treatment means are not equal.
The following R session uses the above equations
to duplicate the Sassafras result for
treatments A and B.

\begin{verbatim}
> YA = c(6,0,2,8,11,4,13,1,8,0)
> YB = c(0,2,3,1,18,4,14,9,1,9)
> sse = var(YA) * (length(YA) - 1) + var(YB) * (length(YB) - 1)
> dfe = length(YA) + length(YB) - 2
> mse = sse / dfe
> se = sqrt(mse * (1 / length(YA) + 1 / length(YB)))
> t = (mean(YA) - mean(YB)) / se
> mean(YA) - mean(YB) - qt(0.975,dfe) * se
[1] -5.922307
> mean(YA) - mean(YB) + qt(0.975,dfe) * se
[1] 4.322307
> 2 * (1 - pt(abs(t),dfe))
[1] 0.746606
\end{verbatim}

The same result is obtained with the t-test function.

\begin{verbatim}
> t.test(YA,YB,var.equal=TRUE)

	Two Sample t-test

data:  YA and YB
t = -0.3281, df = 18, p-value = 0.7466
alternative hypothesis: true difference in means is not equal to 0
95 percent confidence interval:
 -5.922307  4.322307
sample estimates:
mean of x mean of y 
      5.3       6.1 
\end{verbatim}

\end{document}
